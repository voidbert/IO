\documentclass[12pt, a4paper, titlepage]{article}

\usepackage[portuguese]{babel}
\usepackage{float}
\usepackage[a4paper, margin=2cm]{geometry}
\usepackage{graphicx}
\usepackage{setspace}

\chardef\_=`_

\title{\textbf{
    Investigação Operacional -- Trabalho Prático II  \\
    \large Problema de Fluxo Máximo numa Rede
}}
\author{
    \begin{tabular}{ll}
        Ana Carolina Penha Cerqueira       & A104188 \\
        Humberto Gil Azevedo Sampaio Gomes & A104348 \\
        Ivo Filipe Mendes Vieira           & A103999 \\
        José António Fernandes Alves Lopes & A104541 \\
        José Rodrigo Ferreira Matos        & A100612 \\
    \end{tabular}
}
\date{4 de maio de 2024}

\begin{document}

\immediate\write18{mkdir -p resgraph && neato -Tpng DataGraph.dot > resgraph/DataGraph.png}

\onehalfspacing
\setlength{\parskip}{\baselineskip}
\setlength{\parindent}{0pt}
\def\arraystretch{1.5}

\maketitle

\begin{abstract}
    % TODO - fazer abstrato
\end{abstract}

\section{Dados do problema}

Como exigido pelo enunciado, os dados do problema a resolver são determinados em função do número de
aluno mais elevado de entre os integrantes do grupo. No caso, este é A104541, que corresponde ao
aluno José Lopes, e dá origem aos seguintes dados:

\begin{center}
    \begin{tabular}{rl}
        Vértice de origem:  & $O = 6$ \\
        Vértice de destino: & $D = 3$
    \end{tabular}
\end{center}

\begin{table}[H]
    \begin{center}
        \begin{tabular}{c|c}
            Vértice & Capacidade \\
            \hline
            1 & 60        \\
            2 & 90        \\
            3 & $+\infty$ \\
            4 & 50        \\
            5 & 20        \\
            6 & $+\infty$
        \end{tabular}
    \end{center}
    \caption{Capacidade de cada vértice no grafo.}
    \label{vertices-capacities}
\end{table}

\begin{figure}[H]
    \centering
    \includegraphics[width=0.5\textwidth]{resgraph/DataGraph.png}
    \caption{Gráfico no qual deve ser resolvido o problema.}
    \label{data-graph}
\end{figure}

\end{document}
