\documentclass[12pt, a4paper, titlepage]{article}

\usepackage[portuguese]{babel}
\usepackage{caption}
\usepackage{cite}
\usepackage{float}
\usepackage[a4paper, margin=2cm]{geometry}
\usepackage{graphicx}
\usepackage{hyperref}
\usepackage{listings}
\usepackage{setspace}

\chardef\_=`_

\title{\textbf{
    Investigação Operacional -- Trabalho Prático II  \\
    \large Problema de Fluxo Máximo numa Rede
}}
\author{
    \begin{tabular}{ll}
        Ana Carolina Penha Cerqueira       & A104188 \\
        Humberto Gil Azevedo Sampaio Gomes & A104348 \\
        Ivo Filipe Mendes Vieira           & A103999 \\
        José António Fernandes Alves Lopes & A104541 \\
        José Rodrigo Ferreira Matos        & A100612 \\
    \end{tabular}
}
\date{4 de maio de 2024}

\lstdefinestyle{codestyle}{
   basicstyle=\ttfamily\footnotesize,
    breakatwhitespace=false,
    breaklines=false,
    keepspaces=true,
    showspaces=false,
    showstringspaces=false,
    showtabs=false,
    tabsize=4,
}
\lstset{style=codestyle}

\captionsetup{font=onehalfspacing}

\begin{document}

\immediate\write18{neato -Tpng graphs/DataGraph.dot > graphs/DataGraph.png}
\immediate\write18{neato -Tpng graphs/1ConversionDirected.dot > graphs/1ConversionDirected.png}
\immediate\write18{neato -Tpng graphs/2ConversionCapacities.dot > graphs/2ConversionCapacities.png}
\immediate\write18{neato -Tpng graphs/3ConversionMinCost.dot > graphs/3ConversionMinCost.png}
\immediate\write18{neato -Tpng graphs/4ConversionInt.dot > graphs/4ConversionInt.png}

\onehalfspacing
\setlength{\parskip}{\baselineskip}
\setlength{\parindent}{0pt}
\def\arraystretch{1.5}

\maketitle

\begin{abstract}
    % TODO - fazer abstrato
\end{abstract}

\section{Dados do problema}

Como exigido pelo enunciado, os dados do problema a resolver são determinados em função do número de
aluno mais elevado de entre os integrantes do grupo. No caso, este é A104541, que corresponde ao
aluno José Lopes, e dá origem aos seguintes dados:

\begin{center}
    \begin{tabular}{rl}
        Vértice de origem:  & $O = 6$ \\
        Vértice de destino: & $D = 3$
    \end{tabular}
\end{center}

\begin{table}[H]
    \begin{center}
        \begin{tabular}{c|c}
            Vértice & Capacidade \\
            \hline
            1 & 60        \\
            2 & 90        \\
            3 & $+\infty$ \\
            4 & 50        \\
            5 & 20        \\
            6 & $+\infty$
        \end{tabular}
    \end{center}
    \caption{Capacidade de cada vértice no grafo.}
    \label{vertices-capacities}
\end{table}

\begin{figure}[H]
    \centering
    \includegraphics[width=0.5\textwidth]{graphs/DataGraph.png}
    \caption{Grafo no qual deve ser resolvido o problema.}
    \label{data-graph}
\end{figure}

\pagebreak
\section{Formulação do problema}

% TODO - definir problema

Para se utilizar o \emph{solver} RELAX4 \cite{relax} de modo a se encontrar a solução ótima do
modelo, este precisa de ser adaptado para se ter um problema que possa ser dado diretamente ao
\emph{solver}, no caso, um problema de fluxo de custo mínimo numa rede com capacidades em arcos.
Para esta conversão, o grafo original (figura \ref{data-graph}) terá de sofrer alterações, dando
origem a um outro grafo que será providenciado ao RELAX4. Este apresentará uma solução que terá
depois de ser interpretada no grafo original.

O primeiro passo na transformação é a geração de um grafo orientado, onde cada arco $\{i, j\}$ no
grafo original dá origem aos dois arcos orientados $(i, j)$ e $(j, i)$. O resultado desta primeira
transformação pode ser visto na figura abaixo.

\begin{figure}[H]
    \centering
    \includegraphics[width=0.5\textwidth]{graphs/1ConversionDirected.png}
    \caption{Grafo após a aplicação do primeiro passo da transformação.}
    \label{1conversion-graph}
\end{figure}

De seguida, as capacidade dos vértices no modelo original foram convertidas em capacidades de arcos.
Cada vértice $n$ com uma capacidade não infinita (todos exceto a origem e o destino) foi
transformado em dois vértices, $n_e$ e $n_s$, ligados pelo arco $(n_e, n_s)$, de capacidade igual à
do vértice original. Todos os arcos do grafo anterior com destino em $n$ passam a ter destino em
$n_e$, enquanto que os arcos com origem em $n$ passam a ter origem em $n_s$. Assim, todo o fluxo a
passar pelo vértice original passa agora pelo arco de capacidade finita $(n_e, n_s)$, simulando a
capacidade do vértice original. Todos os arcos do grafo original têm capacidade infinita, pelo que
os restantes arcos do novo grafo também o terão. Segue-se o grafo resultante deste segundo passo da
transformação.

\begin{figure}[H]
    \centering
    \includegraphics[width=0.8\textwidth]{graphs/2ConversionCapacities.png}
    \caption{Grafo após a aplicação do segundo passo da transformação. Arcos sem capacidade
        definida têm, por omissão, capacidade infinita.}
    \label{2conversion-graph}
\end{figure}

O passo seguinte trata-se da transformação do problema de fluxo máximo num de fluxo de custo mínimo.
Para tal, considera-se que todos os arcos no grafo anterior têm custo unitário 0. Como o
\emph{solver} procura minimizar o custo e o nosso objetivo é maximizar o fluxo, adiciona-se um arco
entre o destino e a origem, no caso $(6, 3)$, com custo $-1$ \cite{book}. Para tentar minimizar o
custo, o \emph{solver} procurará maximizar o fluxo neste arco, que pode aumentar até não ser
possível aumentar o fluxo entre 3 e 6, calculando-se assim o fluxo máximo. O balanço em cada vértice
é 0, dado que se pretende formar um grafo cíclico onde nenhum fluxo entra ou sai da rede. Mesmo
assim, é possível a existência de fluxo no grafo: o fluxo gerado na origem voltará a este vértice
pelo arco $(3, 6)$, pelo que o balanço na origem será nulo mesmo que haja fluxo no restante grafo.
Segue-se o resultado deste passo da transformação.

\begin{figure}[H]
    \centering
    \includegraphics[width=0.8\textwidth]{graphs/3ConversionMinCost.png}
    \caption{ \onehalfspacing
        Grafo após a aplicação do terceiro passo da transformação. Arcos sem custo definido (todos
        exceto o destacado) têm, por omissão, custo nulo.}
    \label{3conversion-graph}
\end{figure}

O último passo da transformação é uma simples adaptação dos nomes dos vértices e das capacidades dos
arcos para que estes sejam suportados pelo \emph{solver} RELAX4. Em primeiro lugar, todos os nomes
de vértices devem ser convertidos para números naturais, dado que o RELAX4 não suporta vértices com
nomes textuais. De seguida, as capacidades infinitas devem ser convertidas para um valor inteiro
muito grande, consideravelmente superior às restantes capacidades. No caso, utilizámos o valor 1000
para a representação de capacidades infinitas. Segue-se o grafo a ser convertido para um ficheiro de
entrada do RELAX4, onde cada arco é caracterizado por um par contendo, por esta ordem, o custo
unitário do arco e a sua capacidade.

\begin{figure}[H]
    \centering
    \includegraphics[width=0.8\textwidth]{graphs/4ConversionInt.png}
    \caption{Grafo após a transformação do problema.}
    \label{4conversion-graph}
\end{figure}

\section{Ficheiro de entrada do RELAX4}
\lstinputlisting[language={}]{relax4.inp}

\section{Ficheiro de saída do RELAX4}
Foram removidos os comentários do ficheiro de saída que segue.
\lstinputlisting{relax4.out}

\section{Interpretação da solução}

% TODO

% Isto é uma grande gambiarra, mas funciona para pôr a bibliografia como uma secção.
\section{Bibliografia}
\def\refname{}
\vspace{-1.5cm}
\begin{thebibliography}{9}
    \bibitem{relax}
        D. P. Bertsekas and P. Tseng, "RELAX-IV : a faster version of the RELAX code for solving
        minimum cost flow problems,"{} Massachusetts Institute of Technology, Cambridge, MA, USA,
        1994. Accessed: Apr. 14, 2024. Available: \url{https://dspace.mit.edu/handle/1721.1/3392}
    \bibitem{book}
        R. K. Ahuja, T. L. Magnanti, J. B. Orlin, "Introduction,"{} in \emph{Network Flows: Theory,
        Algorithms, and Applications}, 1st ed. Upper Saddle River, NJ, USA: Prentice-Hall, Inc.,
        1993, ch. 1, sec. 2, pp. 4--9.
\end{thebibliography}

\end{document}
