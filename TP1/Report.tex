\documentclass[12pt, a4paper, titlepage]{article}

\usepackage{amsfonts}
\usepackage{amsmath}
\usepackage{amssymb}
\usepackage{array}
\usepackage[portuguese]{babel}
\usepackage{cite}
\usepackage{enumerate}
\usepackage{float}
\usepackage[a4paper, margin=2cm]{geometry}
\usepackage{graphicx}
\usepackage{hyperref}
\usepackage{listings}
\usepackage{lscape}
\usepackage{setspace}
\usepackage[dvipsnames]{xcolor}

\chardef\_=`_

\title{\textbf{
    Investigação Operacional -- Trabalho Prático I  \\
    \large Problema de Empacotamento a Uma Dimensão
}}
\author{
    \begin{tabular}{ll}
        Ana Carolina Penha Cerqueira       & A104188 \\
        Humberto Gil Azevedo Sampaio Gomes & A104348 \\
        Ivo Filipe Mendes Vieira           & A103999 \\
        José António Fernandes Alves Lopes & A104541 \\
        José Rodrigo Ferreira Matos        & A100612 \\
    \end{tabular}
}
\date{23 de março de 2024}

\lstdefinestyle{codestyle}{
    commentstyle=\color{teal},
    keywordstyle=\color{blue},
    numberstyle=\ttfamily\color{gray},
    stringstyle=\color{red},
    basicstyle=\ttfamily\footnotesize,
    breakatwhitespace=false,
    breaklines=false,
    keepspaces=true,
    numbers=left,
    numbersep=10pt,
    showspaces=false,
    showstringspaces=false,
    showtabs=false,
    tabsize=4,
    xleftmargin=3em
}
\lstset{style=codestyle}

\begin{document}

\onehalfspacing
\setlength{\parskip}{\baselineskip}
\setlength{\parindent}{0pt}
\def\arraystretch{1.5}

\maketitle

\begin{abstract}
    Este trabalho prático de Investigação Operacional tem como objetivo a resolução de um problema
    de empacotamento a uma dimensão utilizando o modelo de "um-corte"{}, de Dyckhoff.
    \cite{dyckhoff} Em detalhe, procura-se a formulação do problema, a sua modelação, a sua
    resolução, e a validação do modelo construído. Para tornar o processo de modelação mais rápido e
    menos propício a erros, o nosso grupo implementou conceitos de metaprogramação ao escrever um
    \emph{script} Python que automaticamente gera o código LP do modelo.
\end{abstract}

\section{Dados do problema}

Como exigido pelo enunciado, os dados do problema a resolver são determinados em função do número de
aluno mais elevado de entre os integrantes do grupo. No caso, este é A104541, que corresponde ao
aluno José Lopes, e dá origem aos seguintes dados:

\begin{table}[H]
    \begin{center}
        \begin{tabular}{c|c}
            Capacidade & Quantidade de contentores \\
            \hline
            11          & Ilimitada                 \\
            10          & 5                         \\
            7           & 5
        \end{tabular}
    \end{center}
    \caption{Número de contentores de cada comprimento disponíveis.}
    \label{containers-data}
\end{table}

\begin{table}[H]
    \begin{center}
        \begin{tabular}{c|c}
            Comprimento & Quantidade de itens \\
            \hline
            1           & 0                    \\
            2           & 13                   \\
            3           & 0                    \\
            4           & 9                    \\
            5           & 5
        \end{tabular}
    \end{center}
    \caption{Número de itens de cada comprimento para empacotar.}
    \label{items-data}
\end{table}

A soma dos comprimentos dos itens a empacotar é dada por:

\begin{equation}
    2 \times 13 + 4 \times 9 + 5 \times 5 = 87
    \label{eq:items-sum}
\end{equation}

\section{Formulação do problema}

Pretende-se resolver um problema de empacotamento a uma dimensão, i.e., pretende-se determinar como
se deve distribuir um conjunto de itens por contentores. É necessário ter em conta que deve ser
atribuído exatamente um contentor a cada item, ou seja, um dado item não pode estar em mais do que
um contentor, e não pode haver itens que não tenham um contentor associado. Ademais, a soma dos
comprimentos dos itens em qualquer contentor não pode ultrapassar a sua capacidade.

Neste problema em específico, estão disponíveis contentores de capacidades 11, 10 e 7, sendo que há
contentores de capacidade 11 em número ilimitado, enquanto que há apenas 5 contentores de cada uma
das outras capacidades (tabela \ref{containers-data}). Pretendem-se empacotar 13 itens de
comprimento 2, 9 itens de comprimento 4, e 5 itens de comprimento 5 (tabela \ref{items-data}). O
objetivo é minimizar a soma das capacidade dos contentores utilizados.

\subsection{Modelo de Dyckhoff}

O modelo de "um-corte"{}, de Dyckhoff \cite{dyckhoff}, assenta na ideia do "padrão-um-corte", uma
divisão de um comprimento $k$ em dois menores, $l$ e $k - l$, representada pelo tuplo $[k; l]$.
Note-se que nesta secção adaptaremos o vocabulário do modelo, referente a um problema de
\emph{cutting stock}, para o nosso problema de \emph{bin packing}. Estes cortes são aplicados
recursivamente, começando com valores de $k$ iguais às capacidades dos contentores, e valores de $l$
sempre iguais aos comprimentos dos itens. Os resíduos provenientes do corte $[k; l]$ podem depois
ser divididos em partes menores, e assim sucessivamente.

São denominados resíduos úteis todos os comprimentos que não são inferiores ao mais curto dos itens,
e que podem ser obtidos através de "um-cortes"{} sucessivos a partir das capacidades dos contentores
e dos comprimentos dos itens. Da perspetiva do modelo de Dyckhoff, dois resíduos do mesmo
comprimento são equivalentes, sejam eles capacidades de contentores ou resultados de "um-cortes"{}
anteriores.

Neste modelo, cada variável de decisão $y_{k, l}$ corresponde ao número de comprimentos $k$
divididos de acordo com o "um-corte"{} $[k; l]$. Com base numa solução de um problema por este
modelo, é possível associar os "um-cortes"{} à distribuição dos itens pelos contentores, apesar de
poder haver várias interpretações possíveis para o mesmo conjunto de valores das variáveis.

\bgroup
\renewcommand{\arraystretch}{0}

\begin{figure}[H]
    \centering

    \begin{tabular}{@{}p{1cm}p{1cm}p{1cm}p{1cm}p{1cm}@{}}
        \begin{tabular}{@{}|>{\centering\arraybackslash}m{0.75cm}|}
            \hline \vspace{0.5cm}2\vspace{0.5cm} \\
            \hline \vspace{0.75cm}3\vspace{0.75cm} \\
            \hline \vspace{0.25cm}1\vspace{0.25cm} \\
            \hline
        \end{tabular} &

        \begin{tabular}{@{}|>{\centering\arraybackslash}m{0.75cm}|}
            \hline \vspace{0.5cm}2\vspace{0.5cm} \\
            \hline \vspace{0.5cm}2\vspace{0.5cm} \\
            \hline
        \end{tabular} & &

        \begin{tabular}{@{}|>{\centering\arraybackslash}m{0.75cm}|}
            \hline \vspace{0.5cm}2\vspace{0.5cm} \\
            \hline \vspace{0.5cm}2\vspace{0.5cm} \\
            \hline \vspace{0.5cm}2\vspace{0.5cm} \\
            \hline
        \end{tabular} &

        \begin{tabular}{@{}|>{\centering\arraybackslash}m{0.75cm}|}
            \hline \vspace{0.75cm}3\vspace{0.75cm} \\
            \hline \vspace{0.25cm}1\vspace{0.25cm} \\
            \hline
        \end{tabular}
    \end{tabular}

    \caption{Duas possíveis interpretações para o mesmo conjunto de valores de variáveis de decisão:
             $y_{6,2}=1$, $y_{4,2}=1$ e $y_{4,3}=1$.}
\end{figure}
\egroup

A partir das variáveis de decisão podemos construir o único tipo de restrição que existe no modelo
original de Dyckhoff, denominada de equilíbrio. As restrições deste tipo têm como objetivo assegurar
dois aspetos importantes do modelo. Primeiramente, é exigido que, no fim da aplicação dos
"um-cortes"{}, se tenham produzido espaços suficientes para empacotar todos os itens dentro dos
contentores, tendo cada item um espaço reservado de comprimento igual a si mesmo. Além disso, para
resíduos que não sejam capacidades de contentores, não podem ser cortados mais comprimentos do que
aqueles que foram produzidos a partir da aplicação de "um-cortes". Sendo assim, matematicamente, a
diferença entre os comprimentos produzidos e consumidos através de "um-cortes"{} deve ser superior a
um valor mínimo, o número de itens de um comprimento a empacotar no primeiro caso, ou zero no
segundo.

O objetivo do modelo de "um-corte"{} é o de minimizar o custo dos contentores utilizados. No nosso
caso, o custo de um contentor é a sua capacidade. Logo, a função objetivo é dada pela soma, para
cada tipo de contentor, do produto entre a sua capacidade e o seu número de usos. Trivialmente, o
número de usos de cada tipo de contentor é dado pela diferença entre o número de "um-cortes"{}
executados que consomem o seu comprimento, e o número de "um-cortes"{} que o produzem.

\subsection{Extensões ao modelo de Dyckhoff}

Existem variáveis de decisão que têm o mesmo significado físico ($y_{k, l}$ e $y_{k, k - l}$), como
por exemplo $y_{10,3}$ e $y_{10,7}$. Enquanto que o modelo original não descarta nenhuma variável,
descartar uma das anteriores é possível, e foi como prosseguimos neste trabalho.

Ademais, o modelo de "um-corte"{} não estabelece restrições para a disponibilidade de cada tipo de
contentor, algo necessário para o nosso problema. O cálculo do número de contentores de uma dada
capacidade que foram usados, em função dos "um-cortes"{} executados, já foi descrito na construção
da função objetivo. Basta garantir que este número não é superior à disponibilidade do contentor.

Por último, o modelo de Dyckhoff original considera que todos os resíduos são iguais. Por exemplo,
num problema de corte onde se aplique este modelo, é possível que sobrem resíduos com comprimentos
iguais aos do \emph{stock} a cortar. O modelo considera que o número de rolos gastos destes
comprimentos é negativo, e o valor da função objetivo (custo) diminui, considerando que os rolos
produzidos podem ser guardados para uso posterior. No entanto, isto não faz qualquer sentido num
problema de empacotamento, onde sobras de espaço em contentores não devem ser contabilizadas para a
função objetivo: para cada capacidade de contentor, considera-se o valor máximo entre zero e o custo
dos contentores usados.

\section{Modelo de "um-corte"{} para um problema genérico}

\subsection{Parâmetros e notação utilizada}

Antes de apresentar o modelo final adaptado ao nosso problema, apresentaremos a formulação
matemática do modelo de "um-corte"{} para um caso geral, complementando a descrição meramente
textual feita anteriormente. Alguns objetos matemáticos devem ser conhecidos antes da introdução da
função objetivo e das restrições existentes. Será seguida a mesma notação do artigo que definiu este
modelo. \cite{dyckhoff}

\begin{itemize}
    \item $S \subset \mathbb{\mathbb{Z}^+}$: conjunto dos comprimentos de \emph{stock}, no caso,
        as capacidades dos contentores;
    \item $D \subset \mathbb{\mathbb{Z}^+}$: conjunto dos comprimentos do pedido, no caso, os
        comprimentos dos
        itens;
    \item $R \subset \mathbb{\mathbb{Z}^+}$: conjunto de resíduos relevantes, definidos como
        comprimentos não inferiores ao menor dos itens, que podem ser obtidos através da aplicação
        de "um-cortes"{} sucessivos a partir de $S$ e $D$.
\end{itemize}

Neste modelo, são também dados do problema os seguintes parâmetros:

\begin{itemize}
    \item $N_l \in \mathbb{Z}^+_0$, com $l \in (D \cup R) \setminus S$: número de itens a empacotar
        com o comprimento $l$ (quando $l \not \in D$, $N_l = 0$);
    \item $c_l \in \mathbb{R}$, com $l \in S$: custo de um contentor com comprimento $l$;
    \item $d_l \in \mathbb{Z}^+ \cup \{ \infty \}$, com $l \in S$: disponibilidade de um contentor
        de uma dada capacidade.
\end{itemize}

\subsection{Variáveis de decisão}

Como já explicado, as variáveis de decisão principais deste problema são da forma $y_{k, l}$, e o
seu valor representa o número de divisões de comprimentos $k$ em comprimentos menores, $l$ e
$k - l$. Algumas regras triviais aplicam-se aos valores de $k$ e $l$:

\begin{itemize}
    \item O número de cortes é inteiro e não negativo: $y_{k, l} \in \mathbb{Z}^+_0$;
    \item De um "um-corte"{} não podem resultar comprimentos maiores ou iguais do que o comprimento
        cortado: $l < k$;
    \item De qualquer "um-corte"{}, um dos comprimentos produzidos deve ser o comprimento de um
        item: $l \in D$;
    \item Apenas se podem dividir capacidades de contentores ou de comprimentos obtidos através de
        "um-cortes"{} anteriores (resíduos): $k \in S \cup R$.
\end{itemize}

O conjunto de variáveis no modelo final não é conhecido antes da expansão da função objetivo e das
restrições.

\subsection{Função objetivo}

Na nossa extensão ao modelo de Dyckhoff, são necessárias outras variáveis de decisão para a
linearização da função objetivo. Procura-se minimizar a soma dos custos dos contentores utilizados,
assegurando-se que o número de usos de um dado contentor não é negativo:

\begin{equation}
    \text{min: } \sum_{l \in S} c_l \cdot \max \left \{ 0,
        {\color{blue} \sum_{k \in C_l} y_{l,k}} -
        {\color{ForestGreen} \sum_{k \in B_l} y_{k + l, k}}
    \right \}
    \label{eq:objective}
\end{equation}

Na equação anterior, os conjuntos $B_l$ e $C_l$ assumem os seguintes valores:

\begin{equation}
    B_l = \left \{k \in D \mid k + l \in S \cup R \right \} \label{eq:Bl} \nonumber
\end{equation}

\begin{equation}
    C_l = \left \{k \in D \mid k < l \right \} \label{eq:Cl} \nonumber
\end{equation}

Tendo em conta as definições de $B_l$ e $C_l$, conclui-se que o somatório a {\color{blue} azul}
na função objetivo \eqref{eq:objective} representa o número de comprimentos $l$ consumidos, enquanto
que o somatório a {\color{ForestGreen} verde} representa o número de comprimentos $l$ produzidos
através de "um-cortes"{}.

A linearização desta função objetivo é um processo simples: para cada $l \in S$, considera-se uma
variável de decisão $m_l$, que representa o máximo entre $0$ e a diferença entre os somatórios.
Sendo assim, $m_l$ será superior ou igual a ambos estes valores, e a minimização da soma de todos os
$m_l$ conduzirá a que cada uma destas variáveis assuma exatamente um dos valores ($0$ ou a diferença
entre somatórios), visto que a solução ótima estará na fronteira de um poliedro convexo:

\begin{equation}
    \begin{split}
        \text{min: }      & \sum_{l \in S} c_l \cdot m_l \\
        \forall_{l \in S} & \begin{cases}
            m_l \geq 0, \\
            m_l \geq \sum_{k \in C_l} y_{l,k} - \sum_{k \in B_l} y_{k + l, k}
        \end{cases}
    \end{split}
    \label{eq:objective-expanded}
\end{equation}

\subsection{Restrições}

Ao contrário das restrições anteriores, existentes apenas para ser possível uma representação linear
do modelo, segue-se a apresentação das restrições impostas pelo contexto e dados do problema.

\subsubsection{Restrições de equilíbrio}

Estas restrições são definidas no artigo de Dyckhoff \cite{dyckhoff} e garantem dois aspetos:
produz-se o número de espaços suficientes para empacotar os itens necessários, e não se consomem
mais comprimentos em "um-cortes"{} do que os que são produzidos. Para cada comprimento desejado, ou
resíduo que não igual à capacidade de um contentor ($l \in (D \cup R) \setminus S$), tem-se a
seguinte restrição:

\begin{equation}
    {\color{red} \sum_{k \in A_l} y_{k, l}} +
    {\color{ForestGreen} \sum_{k \in B_l} y_{k + l, k}} -
    {\color{blue} \sum_{k \in C_l} y_{l, k}} \geq N_l
    \label{eq:balance-restriction}
\end{equation}

Os conjuntos $B_l$ e $C_l$ já foram definidos anteriormente, e a definição de $A_l$ segue-se abaixo:

\begin{equation}
    A_l =
    \begin{cases}
        \left \{ k \in S \cup R \mid k > l \right \}, & l \in D \\
        \varnothing, & l \not \in D
    \end{cases}
    \nonumber
\end{equation}

Assim, os somatórios a {\color{red} vermelho} e {\color{ForestGreen} verde} representam o número de
espaços gerados com comprimento $l$, enquanto que o somatório a {\color{blue} azul} representa o
número desses espaços consumidos por "um-cortes"{} subsequentes. O balanço final de cortes não deve
ser inferior a $0$, e nos casos em que há itens a empacotar, $N_l$.

\subsubsection{Restrições de contentores}

Devido à disponibilidade limitada de cada contentor, é necessário estabelecer restrições que
garantam que o número de contentores usado de cada tipo (já calculado na função objetivo) não
ultrapasse a sua disponibilidade. Para cada $l: d_l \not = \infty$, tem-se a seguinte restrição:

\begin{equation}
    {\color{blue} \sum_{k \in C_l} y_{l,k}} -
    {\color{ForestGreen} \sum_{k \in B_l} y_{k + l, k}} \leq d_l
    \label{eq:container-restriction}
\end{equation}

Apesar de denominarmos estas restrições "de contentores"{}, tal foi feito apenas para as distinguir
das do modelo original de Dyckhoff. No entanto, podemos provar que estas se tratam de meras
restrições de equilíbrio que consideram $N_l = -d_l$ (não se pode ultrapassar um dado consumo de
contentores):

\begin{equation}
    \begin{split}
        \nonumber
        {\color{blue} \sum_{k \in C_l} y_{l,k}} -
        {\color{ForestGreen} \sum_{k \in B_l} y_{k + l, k}} \leq d_l
        & \Leftrightarrow \\
        \nonumber
        {\color{red} 0} +
        {\color{ForestGreen} \sum_{k \in B_l} y_{k + l, k}} -
        {\color{blue} \sum_{k \in C_l} y_{l,k}} \geq - d_l
        & \Leftrightarrow_{\color{red} A = \varnothing} \\
        \nonumber
        {\color{red} \sum_{k \in A_l} y_{k, l}} +
        {\color{ForestGreen} \sum_{k \in B_l} y_{k + l, k}} -
        {\color{blue} \sum_{k \in C_l} y_{l,k}} \geq - d_l & = N_l
    \end{split}
\end{equation}

\section{Aplicação do modelo de "um-corte"{} ao nosso problema}

Para os parâmetros do nosso problema, consideramos $S = \{ 7, 10, 11 \}$, as capacidades dos
contentores, $D = \{ 2, 4, 5 \}$, os comprimentos dos itens, e $R = \{ 2, 3, 4, 5, 6, 7, 8, 9 \}$,
os comprimentos dos resíduos úteis, calculados com base na aplicação de "um-cortes"{} sucessivos:

\begin{equation}
    \begin{split}
        R_0           & \leftarrow  \varnothing \\
        [11; 2]: R_1 & \leftarrow R_0 \cup \{ 2, 9 \} \\
        [11; 4]: R_2 & \leftarrow R_1 \cup \{ 4, 7 \} \\
        [11; 5]: R_3 & \leftarrow R_2 \cup \{ 5, 6 \} \\
        [10; 2]: R_4 & \leftarrow R_3 \cup \{ 2, 8 \} \\
                     & \hspace{3mm} \vdots \nonumber
    \end{split} \nonumber
\end{equation}

Dado o número de itens de cada tipo a empacotar, $N_2 = 13$, $N_4 = 9$ e $N_5 = 5$. Ademais, com a
disponibilidade de contentores de cada tipo, $d_7 = d_{10} = 5$ e $d_{11} = \infty$. Sabemos ainda
que cada contentor tem um custo igual à sua capacidade, dado que procuramos minimizar a soma dos
comprimentos dos contentores utilizados: $\forall_{l \in S}, c_l = l$.

Antes da expansão das restrições, não sabemos quais serão as variáveis de decisão que estarão
presentes no modelo final. No entanto, tendo já feito isso, tem-se:

\begin{itemize}
    \item Variáveis necessárias para a linearização da função objetivo ($m_l$ representa o número de
        contentores usados de capacidade $l$): $m_7$, $m_{10}$, $m_{11}$ $\in \mathbb{Z}^+_0$
    \item Variáveis de corte (com $y_{k, l}$ a representar o número de comprimentos $k$ divididos em
        $l$ e $k - l$): $y_{11, 5}$, $y_{11, 4}$, $y_{11, 2}$, $y_{10,5}$, $y_{10,4}$, $y_{10, 2}$,
        $y_{9, 2}$, $y_{7, 5}$, $y_{7, 4}$, $y_{8, 2}$, $y_{6, 4}$, $y_{5, 2}$, $y_{4, 2}$,
        $y_{3, 2}$, $y_{8, 5}$, $y_{9, 5}$, $y_{8, 4}$, $y_{5, 4}$, $y_{6, 5}$ $\in \mathbb{Z}^+_0$
\end{itemize}

A expansão da expressão em \eqref{eq:objective-expanded} resulta na seguinte função objetivo e
restrições:

\begin{equation}
    \begin{split}
        \text{min: }    & 11 m_{11} + 10 m_{10} + 7 m_7 \\
        \text{Suj. a: } & m_{11} \geq   y_{11, 5} + y_{11, 4} + y_{11, 2} \\
                        & m_{10} \geq   y_{10, 5} + y_{10, 4} + y_{10, 2} \\
                        & m_7    \geq - y_{11, 4} - y_{9, 2}  + y_{7, 5} + y_{7, 4} \\
    \end{split} \nonumber
\end{equation}

A expansão das restrições de equilíbrio em \eqref{eq:balance-restriction} resulta nas seguintes
expressões:

\begin{equation}
    \begin{split}
        y_{11, 2} + y_{10, 2} + y_{9, 2} + y_{8, 2} + y_{7, 5} + y_{6, 4} + y_{5, 2} + 2 y_{4, 2} +
            y_{3, 2} & \geq 13 \\
        y_{8, 5} + y_{7, 4} + y_{5, 2} - y_{3, 2} & \geq 0 \\
        y_{11, 4} + y_{10, 4} + y_{9, 5} + 2 y_{8, 4} + y_{7, 4} + y_{6, 4} + y_{5, 4} - y_{4, 2}
            & \geq 9 \\
        y_{11, 5} + 2 y_{10, 5} + y_{9, 5} + y_{8, 5} + y_{7, 5} + y_{6, 5} - y_{5, 4} - y_{5, 2}
            & \geq 5 \\
        y_{11, 5} + y_{10, 4} + y_{8, 2} - y_{6, 5} - y_{6, 4} & \geq 0 \\
        y_{10, 2} - y_{8, 5} - y_{8, 4} - y_{8, 2} & \geq 0 \\
        y_{11, 2} - y_{9, 5} - y_{9, 2} & \geq 0
    \end{split} \nonumber
\end{equation}

Ademais, as restrições relativas ao número máximo de contentores \eqref{eq:container-restriction}
são expandidas para o seguinte:

\begin{equation}
    \begin{split}
        y_{10, 5} + y_{10, 4} + y_{10, 2} & \leq 5 \\
        - y_{11, 4} - y_{9, 2}  + y_{7, 5} + y_{7, 4} & \leq 5
    \end{split}
\end{equation}

É de notar que omitimos as restrições do tipo $m_l, y_{k, l} \geq 0$, pois estas são asseguradas
pelo tipo das variáveis (inteiros não negativos), tanto matematicamente como no \texttt{lpsolve}.
Segue-se o nosso modelo completo:

\begin{equation}
    \begin{split}
        \text{min: }    & 11 m_{11} + 10 m_{10} + 7 m_7 \\
        \text{Suj. a: } & m_{11} \geq   y_{11, 5} + y_{11, 4} + y_{11, 2} \\
                        & m_{10} \geq   y_{10, 5} + y_{10, 4} + y_{10, 2} \\
                        & m_7    \geq - y_{11, 4} - y_{9, 2}  + y_{7, 5} + y_{7, 4} \\ \\
                        & y_{11, 2} + y_{10, 2} + y_{9, 2} + y_{8, 2} + y_{7, 5} + y_{6, 4} +
                          y_{5, 2} + 2 y_{4, 2} + y_{3, 2} \geq 13 \\
                        & y_{8, 5} + y_{7, 4} + y_{5, 2} - y_{3, 2} \geq 0 \\
                        & y_{11, 4} + y_{10, 4} + y_{9, 5} + 2 y_{8, 4} + y_{7, 4} + y_{6, 4} +
                          y_{5, 4} - y_{4, 2} \geq 9 \\
                        & y_{11, 5} + 2 y_{10, 5} + y_{9, 5} + y_{8, 5} + y_{7, 5} + y_{6, 5} -
                          y_{5, 4} - y_{5, 2} \geq 5 \\
                        & y_{11, 5} + y_{10, 4} + y_{8, 2} - y_{6, 5} - y_{6, 4} \geq 0 \\
                        & y_{10, 2} - y_{8, 5} - y_{8, 4} - y_{8, 2} \geq 0 \\
                        & y_{11, 2} - y_{9, 5} - y_{9, 2} \geq 0 \\ \\
                        & y_{10, 5} + y_{10, 4} + y_{10, 2} \leq 5 \\
                        & - y_{11, 4} - y_{9, 2}  + y_{7, 5} + y_{7, 4} \leq 5 \\ \\
                        & y_{l, k}, m_l \in \mathbb{Z}_0^+
    \end{split} \nonumber
\end{equation}

\section{Ficheiro de entrada do lpsolve}
\lstinputlisting[language=c, breaklines=true]{dyckhoff.lp}

\section{Ficheiro de saída do lpsolve}
\lstinputlisting{dyckhoff.out}

\section{Interpretação da solução ótima}

Os valores das variáveis de decisão apresentados na secção acima podem ser associados a uma
distribuição dos itens por contentores. Por observação dos valores das variáveis $m_l$, concluímos
sobre o número de contentores de cada tipo utilizados:

\begin{table}[H]
    \begin{center}
        \begin{tabular}{c|c}
            Capacidade & Contentores utilizados \\
            \hline
            11         & 3 \\
            10         & 4 \\
            7          & 2
        \end{tabular}
    \end{center}
    \caption{Número de contentores de cada tipo utilizados na solução ótima do nosso problema.}
\end{table}

Com base nas variáveis de corte ($y_{k, l}$), podemos agora ir dividindo os comprimentos dos
contentores, executando o número de "um-cortes"{} de cada tipo especificado pela solução. A
distribuição de itens por contentores que deduzimos é a seguinte:

\bgroup
\renewcommand{\arraystretch}{0}

\begin{figure}[H]
    \centering

    \begin{tabular}{p{1cm}p{1cm}p{1cm} p{0.5cm} p{1cm}p{1cm}p{1cm}p{1cm} p{0.5cm} p{1cm}p{1cm}}
        \begin{tabular}{|>{\centering\arraybackslash}m{0.7cm}|}
            \hline \vspace{0.5cm}2\vspace{0.5cm} \\
            \hline \vspace{1.25cm}5\vspace{1.25cm} \\
            \hline \vspace{1cm}4\vspace{1cm} \\
            \hline
        \end{tabular} &
        \begin{tabular}{|>{\centering\arraybackslash}m{0.7cm}|}
            \hline \vspace{0.5cm}2\vspace{0.5cm} \\
            \hline \vspace{1.25cm}5\vspace{1.25cm} \\
            \hline \vspace{1cm}4\vspace{1cm} \\
            \hline
        \end{tabular} &
        \begin{tabular}{|>{\centering\arraybackslash}m{0.7cm}|}
            \hline \vspace{0.5cm}2\vspace{0.5cm} \\
            \hline \vspace{1.25cm}5\vspace{1.25cm} \\
            \hline \vspace{1cm}4\vspace{1cm} \\
            \hline
        \end{tabular} & &

        \begin{tabular}{|>{\centering\arraybackslash}m{0.7cm}|}
            \hline \vspace{1cm}4\vspace{1cm} \\
            \hline \vspace{0.43cm}2\vspace{0.43cm} \\
            \hline \vspace{0.43cm}2\vspace{0.43cm} \\
            \hline \vspace{0.5cm}2\vspace{0.5cm} \\
            \hline
        \end{tabular} &
        \begin{tabular}{|>{\centering\arraybackslash}m{0.7cm}|}
            \hline \vspace{1cm}4\vspace{1cm} \\
            \hline \vspace{0.43cm}2\vspace{0.43cm} \\
            \hline \vspace{0.43cm}2\vspace{0.43cm} \\
            \hline \vspace{0.5cm}2\vspace{0.5cm} \\
            \hline
        \end{tabular} &
        \begin{tabular}{|>{\centering\arraybackslash}m{0.7cm}|}
            \hline \vspace{1cm}4\vspace{1cm} \\
            \hline \vspace{1cm}4\vspace{1cm} \\
            \hline \vspace{0.5cm}2\vspace{0.5cm} \\
            \hline
        \end{tabular} &
        \begin{tabular}{|>{\centering\arraybackslash}m{0.7cm}|}
            \hline \vspace{1cm}4\vspace{1cm} \\
            \hline \vspace{1cm}4\vspace{1cm} \\
            \hline \vspace{0.5cm}2\vspace{0.5cm} \\
            \hline
        \end{tabular} & &

        \begin{tabular}{|>{\centering\arraybackslash}m{0.7cm}|}
            \hline \vspace{0.5cm}2\vspace{0.5cm} \\
            \hline \vspace{1.25cm}5\vspace{1.25cm} \\
            \hline
        \end{tabular} &
        \begin{tabular}{|>{\centering\arraybackslash}m{0.7cm}|}
            \hline \vspace{0.5cm}2\vspace{0.5cm} \\
            \hline \vspace{1.25cm}5\vspace{1.25cm} \\
            \hline
        \end{tabular}
    \end{tabular}

    \caption{Uma distribuição possível dos itens pelos contentores na solução ótima.}
\end{figure}
\egroup

O valor da função objetivo, correspondente à soma dos comprimentos dos contentores utilizados, é
$87$.

\section{Validação do modelo}

Após o cálculo da solução ótima, procedemos à validação do modelo. O primeiro passo consiste em
verificar se a interpretação real da solução obtida faz sentido, e não viola nenhum pressuposto que
possa ter sido esquecido no modelo LP. Verifica-se que:

\begin{itemize}
    \item Todos os itens foram empacotados: 13 de comprimento 12, 9 de comprimento 4 e 5 de
        comprimento 5;
    \item Não foi ultrapassada a capacidade de nenhum dos contentores;
    \item Não foram usados mais contentores do que os disponíveis: apenas 3 de capacidade 10, e 2 de
        capacidade 7, números inferiores aos 5 contentores disponíveis para cada um destes tipos.
\end{itemize}

De seguida, verificamos se o nosso modelo não é demasiado restritivo, i.e., se não existem soluções
melhores do que a atual, válidas no mundo real mas que não respeitam as restrições do modelo LP.
Verificamos que a soma dos comprimentos dos itens (ver equação \ref{eq:items-sum}) é igual à soma
das capacidades dos contentores utilizados, 87, pelo que a nossa solução não gera qualquer espaço
livre, e consequentemente não existe melhor solução possível que não seja considerada pelo nosso
modelo.

\section{Metaprogramação}

A expansão dos vários somatórios do modelo de "um-corte"{} é, devido à sua dimensão, difícil de ser
executada manualmente. O processo não só é demorado, como também propício a erros de cálculo
difíceis de diagnosticar: uma solução errada ou a ausência de solução no modelo final apenas indica
a existência de um erro, mas não a sua localização. Ademais, o processo laborioso de criação do
modelo teria de ser repetido caso os dados iniciais do problema sofressem alterações, uma ocorrência
constante no mundo real. Para endereçar este problema, desenvolvemos \emph{scripts} em Python que
geram o modelo LP automaticamente.

O nosso \emph{script} para a implementação deste modelo encontra-se em anexo (ver
\ref{code:one-cut}) e o seu funcionamento passo a passo é descrito abaixo.

\begin{enumerate}[\hspace{1cm} \bfseries 1.]
    \item Calcular o conjunto de resíduos úteis ($R$), sucessivamente executando todos os cortes
        possíveis com base nas capacidades dos contentores e nos comprimentos dos itens.

    \item Gerar a função objetivo, em particular, a variação que não diminui o custo de uma solução
        caso tenham sobrado espaços com comprimentos no conjunto das capacidades dos contentores
        (ver equação \ref{eq:objective-expanded}).

    \item Gerar as restrições de equilíbrio, calculando os conjuntos e os somatórios descritos em
        \eqref{eq:balance-restriction}.

    \item Gerar restrições a afirmar que o número de contentores de um dado tipo utilizados não pode
        ser superior à disponibilidade dessa capacidade de contentor (ver equação
        \ref{eq:container-restriction}).

    \item Restringir todas as variáveis de corte a valores inteiros.
\end{enumerate}

No nosso \emph{script}, a remoção de variáveis redundantes é feita no processo de simplificação de
adições, tanto na geração da função objetivo como das restrições:
$(y_{10, 2}) + (y_{10, 8} + y_{8, 4})$ será automaticamente transformada em $y_{10, 2} + y_{8, 4}$,
por exemplo.

\section{Conclusão}

Ao longo do desenvolvimento deste trabalho, não só fomos capazes de resolver o problema proposto
com correção, como também obtivemos um profundo conhecimento de um modelo para a resolução de
problemas de empacotamento a uma dimensão. Este conhecimento permitiu-nos implementar programas que
constroem este modelo sem qualquer intervenção humana, tornando a sua aplicação mais simples em
problemas maiores e em contextos em que os dados estão em constante mudança, evitando o desperdício
de horas humanas em cálculos que facilmente podem ser executados por um computador.

% Isto é uma grande gambiarra, mas funciona para pôr a bibliografia como uma secção.
\section{Bibliografia}
\def\refname{}
\vspace{-1.5cm}
\begin{thebibliography}{9}
    \bibitem{dyckhoff}
    H. Dyckhoff, "A New Linear Programming Approach to the Cutting Stock Problem",
    \emph{Operations Research}, vol. 29, no. 6, Dec., pp. 1092-1104, 1981.
    \href{https://doi.org/10.1287/opre.29.6.1092}{doi: 10.1287/opre.29.6.1092}
\end{thebibliography}

\begin{landscape}
    \section{Anexos}

    \subsection{\emph{Script} gerador do modelo de "um-corte"{}}
    \label{code:one-cut}
    \lstinputlisting[language=python]{dyckhoff.py}
    \pagebreak
\end{landscape}

\end{document}
