\documentclass[12pt, a4paper, titlepage]{article}

\usepackage{amsmath}
\usepackage[portuguese]{babel}
\usepackage{cite}
\usepackage{enumerate}
\usepackage[a4paper, margin=2cm]{geometry}
\usepackage{float}
\usepackage{graphicx}
\usepackage{hyperref}
\usepackage{setspace}

\chardef\_=`_

\title{\textbf{
    Investigação Operacional -- Trabalho Prático I  \\
    \large Problema de Empacotamento a Uma Dimensão
}}
\author{
    \begin{tabular}{ll}
        Ana Carolina Penha Cerqueira       & A104188 \\
        Humberto Gil Azevedo Sampaio Gomes & A104348 \\
        Ivo Filipe Mendes Vieira           & A103999 \\
        José António Fernandes Alves Lopes & A104541 \\
        José Rodrigo Ferreira Matos        & A100612 \\
    \end{tabular}
}
\date{23 de março de 2024}

\begin{document}

\onehalfspacing
\setlength{\parskip}{\baselineskip}
\setlength{\parindent}{0pt}
\def\arraystretch{1.5}

\maketitle

\begin{abstract}
    Este trabalho prático de Investigação Operacional tem como objetivo a resolução de um problema
    de empacotamento a uma dimensão utilizando o modelo de "um-corte", de Dyckhoff. \cite{dyckhoff}
    Em detalhe, procura-se a formulação do problema, a sua modelação, a sua resolução, e a validação
    do modelo construído.

	% TODO - adicionar que fizemos meta programação
\end{abstract}

\setcounter{section}{-1}
\section{Dados do problema}

Como exigido pelo enunciado, os dados do problema a resolver são determinados em função do número de
aluno mais elevado de entre os integrantes do grupo. No caso, este é A104541, que corresponde ao
aluno José Lopes, e dá origem aos seguintes dados:

\begin{table}[H]
    \begin{center}
        \begin{tabular}{c|c}
            Capacidade & Quantidade de contentores \\
            \hline
            11          & Ilimitada                 \\
            10          & 5                         \\
            7           & 5
        \end{tabular}
    \end{center}
    \caption{Número de contentores de cada comprimento disponíveis.}
	\label{containers-data}
\end{table}

\begin{table}[H]
    \begin{center}
        \begin{tabular}{c|c}
            Comprimento & Quantidade de itens \\
            \hline
            1           & 0                    \\
            2           & 13                   \\
            3           & 0                    \\
            4           & 9                    \\
            5           & 5
        \end{tabular}
    \end{center}
    \caption{Número de itens de cada comprimento para empacotar.}
	\label{items-data}
\end{table}

A soma dos comprimentos dos itens a empacotar é dada por:

$$2 \times 13 + 4 \times 9 + 5 \times 5 = 87$$

\section{Formulação do problema}

Pretende-se resolver um problema de empacotamento a uma dimensão, i.e., pretende-se determinar como
se deve distribuir um conjunto de itens por contentores. É necessário ter em conta que deve ser
atribuído exatamente um contentor a cada item, ou seja, um dado item não pode estar em mais do que
um contentor, e não pode haver itens que não tenham um contentor associado. Ademais, a soma dos
comprimentos dos itens em qualquer contentor não pode ultrapassar a sua capacidade.

Neste problema em específico, estão disponíveis contentores de capacidades 11, 10 e 7, sendo que há
contentores e capacidade 11 em número ilimitado, enquanto que há apenas 5 contentores de cada uma
das outras capacidades (tabela \ref{containers-data}). Pretendem-se empacotar 13 itens de
comprimento 2, 9 itens de comprimento 4, e 5 itens de comprimento 5 (tabela \ref{items-data}). O
objetivo é minimizar a soma das capacidade dos contentores utilizados.

\section{Modelo de Dyckhoff}

O modelo "um-corte", de Dyckhoff \cite{dyckhoff}, é caracterizado pelo uso dinâmico de um padrão de
corte de estrutura simples, chamado "padrão-um-corte", que descreve o comprimento \textit{l} pelo
qual uma peça, de comprimento \textit{k}, é dividida ([k; l]), sendo \textit{l} inferior a
\textit{k} e pertencente ao conjunto de peças objetivo descrito no parágrafo seguinte.

Este modelo assenta na ideia recursiva por detrás da aplicação subsequente de um número ilimitado de
padrões "um-corte". O padrão vai ser primeiramente aplicado a peças padrão, isto é, peças de um
tamanho \textit{k} que estão à disposição do modelo sem ter sido realizado qualquer corte prévio.
Num processo de corte [k; l] duas peças vão ser geradas, uma de comprimento \textit{l}, e outra de
comprimento \textit{k-l}. O processo de corte pode ser, então, aplicado a ambas as peças geradas
através do processo anterior, além das restantes peças padrão, até que seja formado um conjunto
final de peças objetivo, que atende ao pedido do problema. Devem ser excluídos padrões idênticos,
isto é, padrões em que um corte [k; l] e um corte [k; k-l] geram ambos resíduos idênticos.

Então, cada variável de decisão \textit{$y_{k,l}$} deverá representar a quantidade de peças, cada
uma de um comprimento \textit{k}, pertencentes ao conjunto de peças padrão ou ao conjunto de peças
resíduos, divididas por um qualquer comprimento \textit{l}, pertencente ao conjunto de peças
objetivo e inferior a \textit{k}.

Assumindo a exclusão do caso trivial em que um tamanho \textit{k}, de um conjunto de peças padrão,
corresponde, sem qualquer corte, ao comprimento \textit{l} de um conjunto de peças objetivo, é
facilmente obtido o conjunto de todas as peças resíduo possíveis, originadas a partir do
"padrão-um-corte", e que não têm menor comprimento do que a menor peça objetivo. As peças resíduo
que não se encaixam nesta segunda caracterização pertencem ao conjunto de peças excesso.

A partir das variáveis de decisão, as restrições do modelo vão se formular da seguinte forma: o
número de novas peças de tamanho \textit{l} geradas via um corte, seja ele [k; l] ou [k+l; k]
(\textit{$y_{k,l}$} ou \textit{$y_{k+l,k}$}), em que \textit{k} e \textit{k+l}, respetivamente,
pertencem ao conjunto de peças padrão ou resíduo, deverá ser superior, ou igual, à quantidade de
peças de tamanho \textit{l} utilizadas em processos de corte subsequentes, mais a demanda que é
necessária atender de peças objetivo de tamanho \textit{l}. Não é necessário formular nenhuma
restrição para os excessos.

Se o que procuramos é minimizar o uso de peças padrão, basta, então, minimizar a soma da quantidade
de todas as peças padrão, cada uma de um qualquer comprimento \textit{l}, cortadas por um
comprimento \textit{k} para produzir peças objetivo/resíduo menores. É importante referir que devem
não devem ser contabilizadas peças resíduo, isto é, que foram obtidas a partir de uma peça padrão de
maior comprimento \textit{k+l}.

\section{Metaprogramação}

No modelo de um-corte, a expansão dos sumatórios da função objetivo e das restrições do modelo é,
por vários motivos, difícil de executar manualmente. Devido à grande dimensão dos sumatórios, o
processo não só é demorado, como também propício a erros de cálculo difíceis de diagnosticar: uma
solução errada ou a sua ausência apenas indica a existência de um erro, mas não a sua localização
no modelo. Ademais, a recriação do modelo teria de ser repetida caso os dados iniciais do problema
sofressem alterações, uma ocorrência constante no mundo real. Para endereçar este problema,
desenvolvemos \emph{scripts} em Python que geram o modelo LP automaticamente.

\subsection{Modelo do um-corte}

% TODO - Descrição do modelo dos um-cortes

\subsection{Modelo dos padrões de corte}

Inicialmente, o nosso grupo resolveu o problema de empacotamento proposto utilizando o modelo dos
padrões de corte, de modo a poder verificar a solução do modelo do um-corte quando este fosse
implementado. Também desenvolvemos um \emph{script} que gera este modelo automaticamente, cujo
funcionamento é descrito brevemente:

\begin{enumerate}[\hspace{1cm} \bfseries 1.]
	\item % TODO - descrição do modelo dos padrões de corte
\end{enumerate}

% Isto é uma grande gambiarra, mas funciona para pôr a bibliografia como uma secção.
\section{Bibliografia}
\def\refname{}
\vspace{-1.5cm}
\begin{thebibliography}{9}
    \bibitem{dyckhoff}
    H. Dyckhoff, "A New Linear Programming Approach to the Cutting Stock Problem",
    \emph{Operations Research}, vol. 29, no. 6, Dec., pp. 1092-1104, 1981.
    \href{https://doi.org/10.1287/opre.29.6.1092}{doi: 10.1287/opre.29.6.1092}
\end{thebibliography}

\end{document}
