\documentclass[12pt, a4paper, titlepage]{article}

\usepackage{amsmath}
\usepackage[portuguese]{babel}
\usepackage{cite}
\usepackage{enumerate}
\usepackage{float}
\usepackage[a4paper, margin=2cm]{geometry}
\usepackage{graphicx}
\usepackage{hyperref}
\usepackage{listings}
\usepackage{lscape}
\usepackage{setspace}
\usepackage{xcolor}

\chardef\_=`_

\title{\textbf{
    Investigação Operacional -- Trabalho Prático I  \\
    \large Problema de Empacotamento a Uma Dimensão
}}
\author{
    \begin{tabular}{ll}
        Ana Carolina Penha Cerqueira       & A104188 \\
        Humberto Gil Azevedo Sampaio Gomes & A104348 \\
        Ivo Filipe Mendes Vieira           & A103999 \\
        José António Fernandes Alves Lopes & A104541 \\
        José Rodrigo Ferreira Matos        & A100612 \\
    \end{tabular}
}
\date{23 de março de 2024}

\begin{document}

\onehalfspacing
\setlength{\parskip}{\baselineskip}
\setlength{\parindent}{0pt}
\def\arraystretch{1.5}

\maketitle

\begin{abstract}
    Este trabalho prático de Investigação Operacional tem como objetivo a resolução de um problema
    de empacotamento a uma dimensão utilizando o modelo de "um-corte"{}, de Dyckhoff.
    \cite{dyckhoff} Em detalhe, procura-se a formulação do problema, a sua modelação, a sua
    resolução, e a validação do modelo construído. Para tornar o processo de modelação mais rápido e
    menos propício a erros, o nosso grupo implementou conceitos de metaprogramação ao escrever um
    \emph{script} Python que automaticamente gera o código LP do modelo.
\end{abstract}

\setcounter{section}{-1}
\section{Dados do problema}

Como exigido pelo enunciado, os dados do problema a resolver são determinados em função do número de
aluno mais elevado de entre os integrantes do grupo. No caso, este é A104541, que corresponde ao
aluno José Lopes, e dá origem aos seguintes dados:

\begin{table}[H]
    \begin{center}
        \begin{tabular}{c|c}
            Capacidade & Quantidade de contentores \\
            \hline
            11          & Ilimitada                 \\
            10          & 5                         \\
            7           & 5
        \end{tabular}
    \end{center}
    \caption{Número de contentores de cada comprimento disponíveis.}
    \label{containers-data}
\end{table}

\begin{table}[H]
    \begin{center}
        \begin{tabular}{c|c}
            Comprimento & Quantidade de itens \\
            \hline
            1           & 0                    \\
            2           & 13                   \\
            3           & 0                    \\
            4           & 9                    \\
            5           & 5
        \end{tabular}
    \end{center}
    \caption{Número de itens de cada comprimento para empacotar.}
    \label{items-data}
\end{table}

A soma dos comprimentos dos itens a empacotar é dada por:

$$2 \times 13 + 4 \times 9 + 5 \times 5 = 87$$

\section{Formulação do problema}

Pretende-se resolver um problema de empacotamento a uma dimensão, i.e., pretende-se determinar como
se deve distribuir um conjunto de itens por contentores. É necessário ter em conta que deve ser
atribuído exatamente um contentor a cada item, ou seja, um dado item não pode estar em mais do que
um contentor, e não pode haver itens que não tenham um contentor associado. Ademais, a soma dos
comprimentos dos itens em qualquer contentor não pode ultrapassar a sua capacidade.

Neste problema em específico, estão disponíveis contentores de capacidades 11, 10 e 7, sendo que há
contentores e capacidade 11 em número ilimitado, enquanto que há apenas 5 contentores de cada uma
das outras capacidades (tabela \ref{containers-data}). Pretendem-se empacotar 13 itens de
comprimento 2, 9 itens de comprimento 4, e 5 itens de comprimento 5 (tabela \ref{items-data}). O
objetivo é minimizar a soma das capacidade dos contentores utilizados.

% TODO - Descrever modelo de Dyckhoff

\section{Metaprogramação}

A expansão dos vários sumatórios do modelo de "um-corte"{} é, devido à sua dimensão, difícil de ser
executada manualmente. O processo não só é demorado, como também propício a erros de cálculo
difíceis de diagnosticar: uma solução errada ou a ausência de solução no modelo final apenas indica
a existência de um erro, mas não a sua localização. Ademais, o processo laborioso de criação do
modelo teria de ser repetido caso os dados iniciais do problema sofressem alterações, uma ocorrência
constante no mundo real. Para endereçar este problema, desenvolvemos \emph{scripts} em Python que
geram o modelo LP automaticamente.

\subsection{Modelo de "um-corte"{}}

O nosso \emph{script} para a implementação deste modelo encontra-se em anexo (ver
\ref{code:one-cut}) e o seu funcionamento passo a passo é descrito abaixo.

% TODO - citar fórmulas do modelo matemático neste relatório

\begin{enumerate}[\hspace{1cm} \bfseries 1.]
    \item Calcular o conjunto de resíduos ($R$), sucessivamente executando todos os cortes possíveis
        com base nas capacidades dos contentores e nos comprimentos dos itens.

    \item Gerar a função objetivo, em particular, a variação que não diminui o custo de uma solução
        caso tenham sobrado espaços com comprimentos no conjunto das capacidades dos contentores
        (isto apenas faz sentido em problemas de corte). É necessária a geração de variáveis e
        restrições auxiliares para se calcular o máximo entre $0$ e o custo dos contentores gastos.

    \item Gerar as restrições de equilíbrio, calculando os conjuntos e os sumatórios descritos no
        artigo.

    \item Gerar restrições a afirmar que o número de contentores de um dado tipo utilizados não pode
        ser superior à disponibilidade dessa capacidade de contentor.

    \item Restringir todas as variáveis de corte a valores inteiros.
\end{enumerate}

No nosso \emph{script}, a remoção de variáveis redundantes é feita no processo de simplificação de
adições, tanto na geração da função objetivo como das restrições:
$(y_{10, 2}) + (y_{10, 8} + y_{8, 4})$ será automaticamente transformada em $y_{10, 8} + y_{8, 4}$,
por exemplo.

\subsection{Modelo dos padrões de corte}

Inicialmente, o nosso grupo resolveu o problema de empacotamento proposto utilizando o modelo dos
padrões de corte, de modo a poder verificar a solução do modelo de "um-corte"{} quando este fosse
implementado. Também desenvolvemos um \emph{script} que gera o modelo de Gilmore e Gomory
automaticamente, cujo funcionamento passo a passo é descrito abaixo, e cujo código se encontra em
anexo (ver \ref{code:cutting-patterns}). Como o foco deste trabalho não é o modelo dos padrões de
corte, este não é descrito em detalhe, pelo que é recomendada a leitura do artigo que o definiu.
\cite{gilmore-and-gomory}

\begin{enumerate}[\hspace{1cm} \bfseries 1.]
    \item Calcular os padrões de corte para cada contentor, dado o comprimento de cada item a
        empacotar. Isto é implementado recursivamente: procura-se cortar um comprimento em duas
        partes menores, uma das quais é o comprimento de um item. Calculam-se os padrões de corte da
        outra parte, e usa-se esse resultado para construir os padrões de corte relativos ao
        comprimento total. Um comentário LP é gerado para enumerar todos os padrões de corte
        calculados.

    \item Gerar a função objetivo: conhecem-se todas as variáveis de decisão (padrões de corte),
        cujos coeficientes são as capacidades dos contentores a que estão associadas, dado que se
        pretende minimizar a soma das capacidades dos contentores utilizados.

    \item Calcular quanto cada padrão de corte permite empacotar de cada item, e usar essa
        informação para gerar restrições que obrigam a que pelo menos um dado número de cada item
        seja empacotado.

    \item Gerar restrições relativas ao número máximo de contentores de cada capacidade: a soma de
        todos os padrões de corte usados que envolvem contentores de uma capacidade não pode ser
        superior à disponibilidade desses contentores.

    \item Registar que todas as variáveis correspondentes a padrões de corte são inteiras.
\end{enumerate}

\section{Conclusão}

Ao longo do desenvolvimento deste trabalho, não só fomos capazes de resolver o problema proposto
com correção, como também obtivemos um profundo conhecimento de dois modelos para a resolução de
problemas de empacotamento a uma dimensão. Este conhecimento permitiu-nos implementar programas que
constroem estes modelos sem qualquer intervenção humana, tornando a sua aplicação mais simples em
problemas maiores e em contextos em que os dados estão em constante mudança, evitando o desperdício
de horas humanas em cálculos que facilmente podem ser executados por um computador.

% Isto é uma grande gambiarra, mas funciona para pôr a bibliografia como uma secção.
\section{Bibliografia}
\def\refname{}
\vspace{-1.5cm}
\begin{thebibliography}{9}
    \bibitem{dyckhoff}
    H. Dyckhoff, "A New Linear Programming Approach to the Cutting Stock Problem",
    \emph{Operations Research}, vol. 29, no. 6, Dec., pp. 1092-1104, 1981.
    \href{https://doi.org/10.1287/opre.29.6.1092}{doi: 10.1287/opre.29.6.1092}

    \bibitem{gilmore-and-gomory}
    P. Gilmore, and R. Gomory, "A Linear Programming Approach to the Cutting Stock Problem",
    \emph{Operations Research}, vol. 9, no. 6, Dec., pp. 849-859, 1961.
    \href{https://doi.org/10.1287/opre.9.6.849}{doi: 10.1287/opre.9.6.849}
\end{thebibliography}

\section{Anexos}

\lstdefinestyle{codestyle}{
    commentstyle=\color{teal},
    keywordstyle=\color{blue},
    numberstyle=\ttfamily\color{gray},
    stringstyle=\color{red},
    basicstyle=\ttfamily\footnotesize,
    breakatwhitespace=false,
    breaklines=false,
    keepspaces=true,
    numbers=left,
    numbersep=10pt,
    showspaces=false,
    showstringspaces=false,
    showtabs=false,
    tabsize=4,
    xleftmargin=3em
}
\lstset{style=codestyle}

\begin{landscape}

    \subsection{\emph{Script} gerador do modelo de "um-corte"{}}
\label{code:one-cut}
\lstinputlisting[language=python]{dyckhoff.py}
\pagebreak

\subsection{\emph{Script} gerador do modelo de padrões de corte}
\label{code:cutting-patterns}
\lstinputlisting[language=python]{gilmore_gomory.py}

\end{landscape}

\end{document}
