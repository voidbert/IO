\documentclass[12pt, a4paper, titlepage]{article}

\usepackage{amsmath}
\usepackage{array}
\usepackage[portuguese]{babel}
\usepackage{cite}
\usepackage{enumerate}
\usepackage[a4paper, margin=2cm]{geometry}
\usepackage{float}
\usepackage{graphicx}
\usepackage{hyperref}
\usepackage{setspace}

\chardef\_=`_

\title{\textbf{
    Investigação Operacional -- Trabalho Prático I  \\
    \large Problema de Empacotamento a Uma Dimensão
}}
\author{
    \begin{tabular}{ll}
        Ana Carolina Penha Cerqueira       & A104188 \\
        Humberto Gil Azevedo Sampaio Gomes & A104348 \\
        Ivo Filipe Mendes Vieira           & A103999 \\
        José António Fernandes Alves Lopes & A104541 \\
        José Rodrigo Ferreira Matos        & A100612 \\
    \end{tabular}
}
\date{23 de março de 2024}

\begin{document}

\onehalfspacing
\setlength{\parskip}{\baselineskip}
\setlength{\parindent}{0pt}
\def\arraystretch{1.5}

\maketitle

\begin{abstract}
    Este trabalho prático de Investigação Operacional tem como objetivo a resolução de um problema
    de empacotamento a uma dimensão utilizando o modelo de "um-corte", de Dyckhoff. \cite{dyckhoff}
    Em detalhe, procura-se a formulação do problema, a sua modelação, a sua resolução, e a validação
    do modelo construído.

    % TODO - adicionar que fizemos meta programação
\end{abstract}

\setcounter{section}{-1}
\section{Dados do problema}

Como exigido pelo enunciado, os dados do problema a resolver são determinados em função do número de
aluno mais elevado de entre os integrantes do grupo. No caso, este é A104541, que corresponde ao
aluno José Lopes, e dá origem aos seguintes dados:

\begin{table}[H]
    \begin{center}
        \begin{tabular}{c|c}
            Capacidade & Quantidade de contentores \\
            \hline
            11          & Ilimitada                 \\
            10          & 5                         \\
            7           & 5
        \end{tabular}
    \end{center}
    \caption{Número de contentores de cada comprimento disponíveis.}
    \label{containers-data}
\end{table}

\begin{table}[H]
    \begin{center}
        \begin{tabular}{c|c}
            Comprimento & Quantidade de itens \\
            \hline
            1           & 0                    \\
            2           & 13                   \\
            3           & 0                    \\
            4           & 9                    \\
            5           & 5
        \end{tabular}
    \end{center}
    \caption{Número de itens de cada comprimento para empacotar.}
    \label{items-data}
\end{table}

A soma dos comprimentos dos itens a empacotar é dada por:

$$2 \times 13 + 4 \times 9 + 5 \times 5 = 87$$

\section{Formulação do problema}

Pretende-se resolver um problema de empacotamento a uma dimensão, i.e., pretende-se determinar como
se deve distribuir um conjunto de itens por contentores. É necessário ter em conta que deve ser
atribuído exatamente um contentor a cada item, ou seja, um dado item não pode estar em mais do que
um contentor, e não pode haver itens que não tenham um contentor associado. Ademais, a soma dos
comprimentos dos itens em qualquer contentor não pode ultrapassar a sua capacidade.

Neste problema em específico, estão disponíveis contentores de capacidades 11, 10 e 7, sendo que há
contentores e capacidade 11 em número ilimitado, enquanto que há apenas 5 contentores de cada uma
das outras capacidades (tabela \ref{containers-data}). Pretendem-se empacotar 13 itens de
comprimento 2, 9 itens de comprimento 4, e 5 itens de comprimento 5 (tabela \ref{items-data}). O
objetivo é minimizar a soma das capacidade dos contentores utilizados.

\subsection{Modelo de Dyckhoff}

O modelo de "um-corte"{}, de Dyckhoff \cite{dyckhoff}, assenta na ideia do "padrão-um-corte", uma
divisão de um comprimento $k$ em dois menores, $l$ e $k - l$, representada pelo tuplo $[k; l]$.
Note-se que nesta secção adaptaremos o vocabulário do modelo, referente a um problema de
\emph{cutting stock}, para o nosso problema de \emph{bin packing}. Estes cortes são aplicados
recursivamente, começando com valores de $k$ iguais às capacidades dos contentores, e valores de $l$
sempre iguais aos comprimentos dos itens. Os resíduos provenientes do corte $[k; l]$ podem depois
ser divididos em partes menores, e assim sucessivamente.

São denominados resíduos todos os comprimentos que não são inferiores ao mais curto dos itens, e
que podem ser obtidos através de "um-cortes"{} sucessivos a partir das capacidades dos contentores e
dos comprimentos dos itens. Da perspetiva do modelo de Dyckhoff, dois resíduos do mesmo comprimento
são equivalentes, sejam eles capacidades de contentores ou resultados de "um-cortes"{} anteriores.

Neste modelo, cada variável de decisão $y_{k, l}$ corresponde ao número de comprimentos $k$
divididos de acordo com o "um-corte"{} $[k; l]$. Com base numa solução de um problema por este
modelo, é possível associar os "um-cortes"{} à distribuição dos itens pelos contentores, apesar de
poder haver várias interpretações possíveis para o mesmo conjunto de valores das variáveis.

\bgroup
\renewcommand{\arraystretch}{0}

\begin{figure}[H]
    \centering

    \begin{tabular}{@{}p{2cm}p{2cm}p{1cm}p{2cm}p{2cm}@{}}
        \begin{tabular}{@{}|>{\centering\arraybackslash}m{1.5cm}|}
            \hline \vspace{0.5cm}2\vspace{0.5cm} \\
            \hline \vspace{0.75cm}3\vspace{0.75cm} \\
            \hline \vspace{0.25cm}1\vspace{0.25cm} \\
            \hline
        \end{tabular} &

        \begin{tabular}{@{}|>{\centering\arraybackslash}m{1.5cm}|}
            \hline \vspace{0.5cm}2\vspace{0.5cm} \\
            \hline \vspace{0.5cm}2\vspace{0.5cm} \\
            \hline
        \end{tabular} & &

        \begin{tabular}{@{}|>{\centering\arraybackslash}m{1.5cm}|}
            \hline \vspace{0.5cm}2\vspace{0.5cm} \\
            \hline \vspace{0.5cm}2\vspace{0.5cm} \\
            \hline \vspace{0.5cm}2\vspace{0.5cm} \\
            \hline
        \end{tabular} &

        \begin{tabular}{@{}|>{\centering\arraybackslash}m{1.5cm}|}
            \hline \vspace{0.75cm}3\vspace{0.75cm} \\
            \hline \vspace{0.25cm}1\vspace{0.25cm} \\
            \hline
        \end{tabular}
    \end{tabular}

    \caption{Duas possíveis interpretações para o mesmo conjunto de valores de variáveis de decisão:
             $y_{6,2}=1$, $y_{4,2}=1$ e $y_{4,3}=1$.}
\end{figure}
\egroup

A partir das variáveis de decisão podemos construír o único tipo de restrição que existe no modelo
original de Dyckhoff, denomidada de equilíbrio. As restrições deste tipo têm como objetivo assegurar
dois aspetos importantes do modelo. Primeiramente, é exigido que, no fim da aplicação dos
"um-cortes"{}, se tenham produzido espaços suficientes para empacotar todos os itens dentro dos
contentores, tendo cada item um espaço reservado de comprimento igual a si mesmo. Além disso, para
resíduos que não sejam capacidades de contentores, não podem ser cortados mais comprimentos do que
aqueles que foram produzidos a partir da aplicação de "um-cortes". Sendo assim, matematicamente, a
diferença entre os comprimentos produzidos e consumidos através de "um-cortes"{} deve ser superior a
um valor mínimo, o número de itens de um comprimento a empacotar no primeiro caso, ou zero no
segundo.

O objetivo do modelo de "um-corte"{} é o de minimizar o custo dos contentores utilizados. No nosso
caso, o custo de um contentor é a sua capacidade. Logo, a função objetivo é dada pela soma, para
cada tipo de contentor, do produto entre a sua capacidade e o seu número de usos. Trivialmente, o
número de usos de cada tipo de contentor é dado pela diferença entre o número de "um-cortes"{}
executados que consomem o seu comprimento, e o número de "um-cortes"{} que o produzem.

\subsection{Extensões ao modelo de Dyckhoff}

Existem variáveis de decisão que têm o mesmo significado físico ($y_{k, l}$ e $y_{k, k - l}$), como
por exemplo $y_{10,3}$ e $y_{10,7}$. Enquanto que o modelo original não descarta nenhuma variável,
descartar uma das anteriores é possível, e foi como prosseguimos neste trabalho.

Ademais, o modelo de "um-corte"{} não estabelece restrições para a disponibilidade de cada tipo de
contentor, algo necessário para o nosso problema. O cálculo do número de contentores de uma dada
capacidade que foram usados, em função dos "um-cortes"{} executados, já foi descrito na construção
da função objetivo. Basta garantir que este número não seja superior à disponibilidade do contentor.

Por último, o modelo de Dyckhoff original considera que todos os resíduos são iguais. Por exemplo,
num problema de corte onde se aplique este modelo, é possível que sobrem resíduos com comprimentos
iguais aos do \emph{stock} a cortar. O modelo considera que o número de rolos gastos destes
comprimentos é negativo, e o valor da função objetivo (custo) diminui, considerando que os rolos
produzidos podem ser guardados para uso posterior. No entanto, isto não faz qualquer sentido num
problema de empacotamento, onde sobras de espaço em contentores não devem ser contabilizadas para a
função objetivo: para cada capacidade de contentor, considera-se o valor máximo entre zero e o custo
dos contentores usados.

\section{Metaprogramação}

No modelo de um-corte, a expansão dos sumatórios da função objetivo e das restrições do modelo é,
por vários motivos, difícil de executar manualmente. Devido à grande dimensão dos sumatórios, o
processo não só é demorado, como também propício a erros de cálculo difíceis de diagnosticar: uma
solução errada ou a sua ausência apenas indica a existência de um erro, mas não a sua localização
no modelo. Ademais, a recriação do modelo teria de ser repetida caso os dados iniciais do problema
sofressem alterações, uma ocorrência constante no mundo real. Para endereçar este problema,
desenvolvemos \emph{scripts} em Python que geram o modelo LP automaticamente.

\subsection{Modelo do um-corte}

% TODO - Descrição do modelo dos um-cortes

\subsection{Modelo dos padrões de corte}

Inicialmente, o nosso grupo resolveu o problema de empacotamento proposto utilizando o modelo dos
padrões de corte, de modo a poder verificar a solução do modelo do um-corte quando este fosse
implementado. Também desenvolvemos um \emph{script} que gera este modelo automaticamente, cujo
funcionamento é descrito brevemente:

\begin{enumerate}[\hspace{1cm} \bfseries 1.]
    \item % TODO - descrição do modelo dos padrões de corte
\end{enumerate}

% Isto é uma grande gambiarra, mas funciona para pôr a bibliografia como uma secção.
\section{Bibliografia}
\def\refname{}
\vspace{-1.5cm}
\begin{thebibliography}{9}
    \bibitem{dyckhoff}
    H. Dyckhoff, "A New Linear Programming Approach to the Cutting Stock Problem",
    \emph{Operations Research}, vol. 29, no. 6, Dec., pp. 1092-1104, 1981.
    \href{https://doi.org/10.1287/opre.29.6.1092}{doi: 10.1287/opre.29.6.1092}
\end{thebibliography}

\end{document}
